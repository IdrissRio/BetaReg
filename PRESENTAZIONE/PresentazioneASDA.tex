\documentclass[usenames,dvipsnames]{beamer}
\usetheme[
%%% option passed to the outer theme
    progressstyle=movingCircCnt,   % fixedCircCnt,  (moving is deault)
  ]{Feather}
\setbeamercolor{Feather}{fg=black!20,bg=black}
\setbeamercolor{structure}{fg=black}
\usepackage{graphicx}
%\usepackage[utf8]{inputenc}
\usepackage{amsmath}
\usepackage[italian]{babel} 
\usepackage{tabularx}
\usepackage[T1]{fontenc} 
\usepackage{listings}
\lstset{% setup listings
	language=R,% set programming language
	basicstyle=\ttfamily\footnotesize,% basic font style
	keywordstyle=\color{blue},% keyword style
	commentstyle=\color{gray},% comment style
	numbers=none,% display line numbers on the left side
	numberstyle=\scriptsize,% use small line numbers
	numbersep=10pt,% space between line numbers and code
	backgroundcolor = \color{yellow!10},
	framexleftmargin = 1em,
	framextopmargin = 1em
	framexbottommargin = 1em
	tabsize=3,% sizes of tabs
	showstringspaces=false,% do not replace spaces in strings by a certain character
	captionpos=b,% positioning of the caption below
	breaklines=true% automatic line breaking
}

% colored hyperlinks
\newcommand{\chref}[2]{
  \href{#1}{{\usebeamercolor[bg]{Feather}#2}}
}
\newcommand{\cfbox}[2]{%
	\colorlet{currentcolor}{.}%
	{\color{#1}%
		\fbox{\color{currentcolor}#2}}%
}




\title[Un pacchetto R: betareg] % [] is optional - is placed on the bottom of the sidebar on every slide
{ % is placed on the title page
      \textbf{Un pacchetto R:\\ BETAREG}
}

\subtitle[ | BETAREG]
{
      \textbf{Applied statistics and data analysis}
}

\author[Riouak Idriss, Marta Rotari]
{      Riouak Idriss\\Marta Rotari
}

\institute[]
{
      
    Università degli studi di Udine\\
    Dipartimento di Matematica e Informatica \\
    Laurea specialistica in Informatica\\
  
 
}

\date{9 Gennaio 2018}

%-------------------------------------------------------
% THE BODY OF THE PRESENTATION
%-------------------------------------------------------

\begin{document}

%-------------------------------------------------------
% THE TITLEPAGE
%-------------------------------------------------------

{\1
\begin{frame}[plain,noframenumbering] 
  \titlepage 
\end{frame}}

\begin{frame}{Riferimenti Bibliografici}
\setbeamertemplate{bibliography item}[article]
\begin{thebibliography}{99}
	\bibitem{AaronBradleyZhoar} Cribari-Neto Francisco, Achim Zeileis \textit{Beta Regression in R} (2006).
\bibitem{AaronBradley} Ferrari SLP, Cribari-Neto Francisco \textit{Beta Regression for Modelling Rates and Proportions} (2004). 

\bibitem{ABradleyCimatti} Simas AB, BarretoSouza W, Rocha AV \textit{Improved Estimators for a General Class of Beta Regression Models} (2010) 
\bibitem{Griggio} Paolo Vidoni \textit{Towards multiple linear regression and logistic regression}  2017-2018. Lecture 5. Applied Statistics and Data Analysis.
\end{thebibliography}
\end{frame}
\section{Definizioni e Notazioni}
\begin{frame}[fragile]{Introduzione: modello di regressione}
\begin{itemize}[<+->]
	\item[] Un modello di regressione ha lo scopo di studiare le relazioni tra una variabile $(y)$ detta variabile di \alert{risposta}, e una o più variabili  \alert{regressori} $(x)$. Inoltre permette di effettuare predizioni dato un nuovo valore per la variabile $x$.
	
	\item[]	Nel 2004  viene definito il \alert{Modello di regressione con variabile risposta Beta} [2] per variabili continue con valori in (0, 1), come possono essere \emph{proporzioni} e \emph{ tassi}, assumendo che la variabile risposta sia beta-distribuita. \item[] Successivamente nel 2006, nell'articolo <<Beta Regression in R>> [1] ne viene fornita un'implementazione in R.
\end{itemize}


\end{frame}

\begin{frame}{Distribuzione Beta}
\begin{itemize}
\item[]La variabile di risposta $y$ ha una funzione di distribuzione di probabilità continua definita da due parametri sull'intervallo $(0,1)$. 
$$f(y;\mu, \phi )=\frac{\Gamma (\phi )}{\Gamma (\mu \phi )\Gamma ((1-\mu )\phi )}y^{\mu \phi -1}(1-y)^{(1-\mu)\phi-1}$$
con $0<y<1$ dove $0<\mu<1$, $\phi>0$ e $\Gamma(z)=\int_{0}^{+\infty}t^{z-1}e^{-t}dt.$
\item[]Denoteremo $y \sim \mathcal{B}(\mu,\phi)$ la variabile casuale con parametri $\mu \ e \ \phi$ (detto parametro di precisione): $$E(y)=\mu \ e \ V(y)=\frac{\mu (1-\mu )}{1+\phi}$$
\end{itemize}
\end{frame}

\begin{frame}
\vspace{1.5cm}
\begin{figure}[ht]
\centering
\includegraphics[scale=.22]{BetaDistribution}
\caption{Rappresentazione grafica della distribuzione Beta}
\end{figure}


\end{frame}




\begin{frame}{Modello di regressione Beta}
Sia $y_1,y_2,...,y_n$ un \textbf{campione casuale} tale che $y_i \sim \mathcal{B}(\mu_i,\phi)$ 
il modello di regressione con variabile di risposta \emph{beta distribuita} è
$$g(\mu_i)=x_i ^t \beta=\eta_i $$
\begin{itemize}[<+->]
	\item $\beta=(\beta_1, \beta_2, ..., \beta_k)^t$, con $k<n$, vettore dei coefficienti,
	\item $x_i=(x_{i1}, x_{i2},...,x_{ik})^t$, per convenzione $x_{i1}=1$,
	\item $\eta_i=\beta_1x_{i1}+...+\beta_kx_{ik}$ predittore lineare,
	\item $g(\cdot):(0,1)\rightarrow\mathbb{R}$ è una funzione $\mathcal{C}^2$, detta funzione di collegamento, avente derivata seconda costante.
\end{itemize}
\end{frame}

\begin{frame}{Funzioni di collegamento}
\vspace{0.5cm}
\hspace{0.5cm}
Le funzioni di collegamento più utilizzate sono:
\begin{itemize}
\item \textbf{logit:} $g(\mu)=log(\frac{\mu}{(1-\mu)}) $
\item \textbf{probit:} $g(\mu)=\varPhi^{-1}(\mu)$, dove $\varPhi(\cdot)$ è la funzioni di distribuzione normale standard.
\item \textbf{log-log complementare:}\\ $g(\mu)=\log(-\log(1-\mu))$
\item \textbf{log-log:} $g(\mu)=\log(-\log(\mu))$
\item \textbf{Cauchy:} $g(\mu)=\tan(\pi(\mu-0.5))$
\end{itemize}
\end{frame}

\begin{frame}{Dispersione}
\begin{itemize}
\item[] Nel 2010 è stata formulata un'estensione del modello in
modello di regressione beta a dispersione variabile che considera il parametro di precisione non più costante.
\item[] Le osservazioni $y_i \sim \mathbb{\mathcal{B}}(\mu_i, \phi_i)$ indipendenti con 
\begin{align*}
g_1(\mu_i)&=\eta_{1i}=x{_i}^T \beta,\\ %\label{eq:mean}\\
g_2(\phi_i)&=\eta_{2i}=z{_i}^T \gamma,% \label{eq:precisione}
\end{align*}
\item[] dove $\beta=(\beta_1,....,\beta_k)^T  \  e \ \gamma=(\gamma_1,..,\gamma_h)^T \ con \  k+h<n$ insiemi dei coefficienti di regressione, $x_i \ e \ z_i$ vettori di reggressori e $\eta_{1i} \ e \ \eta_{2i} $ predittori lineari. 
\end{itemize}
\end{frame}

\begin{frame}[fragile]{Pacchetto: betareg}
\begin{itemize}
	\item Il pacchetto \textbf{betareg} è una \textbf{collezione} di funzioni implementate in ``R ",  con il quale è possibile modellare variabili dipendenti beta distribuite.
	\item Le versioni dalla 1.0 alla 1.2 sono state implementate da \emph{Simas e Rocha} fino al 2006. Dalla versione 2.0, il principale contribuente è stato \emph{Achim Zeileis} coautore dell'articolo
	\item 	La main-function \texttt{betareg()} è stata progettata e implementata per essere il più simile possibile alla funzione standard \texttt{glm()} (General Linear Model).
	
\end{itemize}
 


\end{frame}

\begin{frame}[fragile]{Signature della funzione}
\vspace{-0.5cm}
\begin{lstlisting}[,label={lst:signature}]
betareg(formula, data, subset, na.action, weights, 
offset, link = "logit", link.phi = NULL, control = betareg.control(...), model = TRUE, y = TRUE, x = FALSE, ...)
\end{lstlisting}
\begin{itemize}[<+->]
	\item \texttt{formula}: descrizione simbolica del modello, e.g.: \texttt{y $\sim$ x+z}
	\item \texttt{data, subset:} sorgente dei dati.
	\item \texttt{na.action:} funzione che descrive come comportarsi dinnanzi a elementi \texttt{NA}.
	\item \texttt{weights}: vettore di pesi. 
	\item \texttt{link:} specifica la funzione di collegamento. Il valore di default è \texttt{logit}. Le possibili scelte sono: \texttt{probit, cloglog, cauchit, log, loglog}
\end{itemize}

\end{frame}


\begin{frame}[fragile]{Signature della funzione}

\begin{lstlisting}[,label={lst:signature}]
betareg(formula, data, subset, na.action, weights, 
offset, link = "logit", link.phi = NULL, control = betareg.control(...), model = TRUE, y = TRUE, x = FALSE, ...)
\end{lstlisting}
\begin{itemize}[<+->]
\item \texttt{link.phi}: specifica la funzione di collegamento per il parametro di precisione $\phi$  La scelte possibili sono: 
		\begin{itemize}
			\item sqrt
			\item log
		\end{itemize}
\item \texttt{control}: prende come parametro un oggetto di tipo \texttt{betareg.control}.
\item \texttt{model, x, y}: argomenti di tipo logico (\texttt{TRUE, FALSE}). Se impostati a  \texttt{TRUE}, vengono restituiti il \emph{model.frame, model matrix} e il \emph{vettore della variabile risposta} rispettivamente.
\end{itemize}	
\end{frame}

\begin{frame}{Esempi su Shiny}
\vspace{0cm}

\begin{center}
	\href{http://127.0.0.1:5912}{Esempi su Shiny}\\
	\href{http://127.0.0.1:3491}{\includegraphics[scale=0.2]{Shiny}}

\end{center}
\end{frame}

\begin{frame}[fragile, noframenumbering]{Oggetto betareg.control}

Lopzione \texttt{control} prende come parametro un oggetto di tipo \texttt{betareg.control}. Tali oggetti servono per controllare la modalità con la quale veogno stimati i coefficienti del modello.
\begin{lstlisting}[,label={lst:signature}]
betareg.control(phi = TRUE, method = "BFGS", maxit = 5000, hessian = FALSE, trace = FALSE, start = NULL, fsmaxit = 200, fstol = 1e-8, ...)
\end{lstlisting}
\begin{itemize}
	\item \texttt{phi}: valore booleano. Indica se il parametro $\phi$ deve essere trattato come un parametro del modello o come un parametro di disturbo.
	\item \texttt{method}: specifica quale metodo numerico viene utilizzato per stimare i coefficienti. Il valore di default è \texttt{BFGS}.
	\item \texttt{maxit}: indica il numero massimo di iterate da eseguire.
\end{itemize}
\end{frame}



\begin{frame}[fragile,noframenumbering]{Oggetto betareg.control}

\begin{itemize}
		\item \texttt{trace:} valore booleno. Indica se deve essere mantenuta traccia delle iterazioni effettuate durante la stima dei coefficienti.
		\item \texttt{hessian:} valore booleano. Indica se deve essere utilizzato l'Hessiano per stimare la matrice delle covariate. L'opzione di default è \texttt{FALSE}.
		\item \texttt{start:} un vettore di valori opzionali che indica quali sono in punti di partenza per stimare i coefficienti del modello. Si veda la sezione \texttt{III.i}.
		\item \texttt{fsmaxit:} valore intero. Indica il numero massimo delle iterazioni del \textit{punteggio di Fisher}.
		\item \texttt{fstol:} valore numerico. Indica la tolleranza dell'errore per raggiungere la convergenza.
	\end{itemize}
	
\end{frame}


\begin{frame}[noframenumbering]{Metodi classe betareg}

L'oggetto restituito dal modello stimato della classe \texttt{betareg} è una lista simile a quella restituita dagli oggetti \texttt{glm}, dove troviamo \texttt{coefficients} e \texttt{terms}. E' possibile interrogare gli oggetti della classe \texttt{betareg} attraverso la funzione \texttt{summary()}. Nella seguente tabella è stata riporta una lista dei metodi e delle funzioni offerte dagli oggetti della classe \texttt{betareg}. 

\begin{table}[t]
	\centering
	\label{MethodTable}
	\begin{tabularx}{\textwidth}{|l|X|}
		\hline
		\textbf{FUNZIONE}  & \textbf{DESCRIZIONE}               \\ \hline
		\texttt{print()}            & Stampa su standard output i coefficienti stimati.                          \\
		\texttt{summary()}          & Stampa su standard output la stima dei coefficienti, l'errore standard e il test parziale di Wald. Restituisce un oggetto della classe \texttt{summary.betareg}.                        \\ \hline
			\end{tabularx}
		\end{table}
\end{frame}

\begin{frame}[noframenumbering]{Metodi classe betareg}

\begin{table}[t]
	\centering
	%	\caption{Principali funzioni e metodi per gli oggetti della classe \texttt{betareg}}
	\label{MethodTable}
	\begin{tabularx}{\textwidth}{|l|X|}
		\hline
				
				\texttt{coef()}             & Restituisce tutti i coefficienti del modello, compresi intercetta e coefficiente di  precisione.                      \\
				\texttt{vcoef()}            & Restituisce la matrice della covarianza.                        \\
				\texttt{predict()}          & Funzione di predizione del valor atteso, dei predittori lineari, del parametro di precisione e delle varianze.                        \\
				\texttt{fitted()}           & Valori attesi stimati per un nuovo vettore di osservazioni.                 \\
				\texttt{residual()}         & Restituisce il vettore dei residui.                    \\
				\texttt{terms()}            & Restituisce i componenti del modello.       \\
				\texttt{model.matrix()}     & Restituisce la \emph{Matrice del Modello} \\
				\texttt{model.frame()}      & Restituisce l'intero frame di dati del modello.                       \\
				\texttt{loglik()}           & Restituisce la stima di log-verosimiglianza.                    \\ \hline
	\end{tabularx}
\end{table}
\end{frame}

\begin{frame}[noframenumbering]{Metodi classe betareg}
	\vspace{1cm}
	\begin{table}[t]
		\centering
		%	\caption{Principali funzioni e metodi per gli oggetti della classe \texttt{betareg}}
		\label{MethodTable}
		\begin{tabularx}{\textwidth}{|l|X|}
			\hline
		\texttt{plot()}             & Stampa sul device grafico i plot dei residui, delle predizioni, dei punti di leva etc.                            \\
		\texttt{hatvalues}          & Restituisce un vettore di elementi rappresentanti la diagonale della matrice \emph{hat}.                     \\
		\texttt{cocks.distance}     & Restituisce un'approssimazione della distanza di Cook.                     \\
		\texttt{glevarage()}        & Restituisce un vettore di elementi con rappresentanti il valore di leva di ogni punto.                      \\ \hline
		\end{tabularx}
	\end{table}
\end{frame}

\begin{frame}[noframenumbering]{Metodi classe betareg}
	\vspace{1cm}
	\begin{table}[t]
		\centering
		%	\caption{Principali funzioni e metodi per gli oggetti della classe \texttt{betareg}}
		\label{MethodTable}
		\begin{tabularx}{\textwidth}{|l|X|}
			\hline
			\texttt{coeftest()}         & Test parziale di Wald dei coefficienti.                             \\
			\texttt{waldtest()}        & Test di Wald per modelli annidati.                      \\
			\texttt{linear.hypotesis()} &  Test di Wald per ipotesi lineari.                \\
			\texttt{AIC()}           & Calcola l'Aikaike Information Criteria (AIC) e altri Information Criteria come il BIC.                      \\ \hline
		\end{tabularx}
	\end{table}
\end{frame}

\begin{frame}[fragile,noframenumbering]{Funzione Predict}
\begin{lstlisting}[,label={lst:signature}]
predict(object, newdata = NULL, type = c("response", "link", "precision", "variance", "quantile"), na.action = na.pass, at = 0.5, ...)
\end{lstlisting}
\begin{itemize}
	\item object: fitted model object della classe betareg
	\item newdata: opzionale, un data frame nel quale cercare delle variabili con le quali fare predizione. Se omesso, vengono usate le osservazioni originali
	\item type: character che indica il tipo di predizione: 
	\begin{itemize}
		\item \texttt{response}: fitted means of response, 
		\item \texttt{link}:corresponding linear predictor, 
		\item \texttt{precision}fitted precision parameter phi, 
		\item \texttt{variance}fitted variances of response, 
		\item \texttt{quantile}fitted quantile(s) of the response distribution 
	\end{itemize}

\end{itemize}
\end{frame}

\begin{frame}[fragile,noframenumbering]{Funzione Predict}
\begin{itemize}
\item at: vettore numerico che indica il livello al quale i quantili vengono predetti se nel parametro
textbf{type="quantile"} di default è assegnata la mediana \textbf{at=0.5}
\end{itemize}
 \begin{lstlisting}[,label={lst:signature}]
 data("GasolineYield", package = "betareg")
 gy2 <- betareg(yield ~ batch + temp | temp, data = GasolineYield)
 cbind(predict(gy2, type = "response"),
 predict(gy2, type = "link"),
 predict(gy2, type = "precision"),
 predict(gy2, type = "variance"),
 predict(gy2, type = "quantile", at = c(0.25, 0.5, 0.75)))
 \end{lstlisting}
\end{frame}



\begin{frame}[noframenumbering]{Intervallo di confidenza}
	E' possibile determinare un intervallo di confidenza $(1-\alpha)100\%$ per i coefficienti $\hat{\beta}_j$, con $j=1,...,k$. Tale intervallo è:
	$$ \bigg[\hat{\beta_j} \pm \Phi^{-1}\bigg(1-\frac{\alpha}{2}SE(\hat{\beta_j})\bigg)\bigg], $$
	dove $\Phi(\cdot)$ è la f.d.p di una normale standard. 
	
	Analogamente per il parametro $\hat{\phi}$ è il seguente
	$$\bigg[\hat{\phi} \pm \Phi^{-1}\bigg(1-\frac{\alpha}{2}SE(\hat{\phi})\bigg)  \bigg] $$ 
	dove $SE(\hat{\phi})=\sqrt{tr(D)-\phi^{-1}c^tT^tX(X^tWX)^{-1}X^tTc}=\sqrt{\hat{\gamma}}.$

\end{frame}

\end{document}
